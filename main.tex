%% https://tex.stackexchange.com/questions/366229/an-aesthetically-pleasing-recipe-book-template
\documentclass{article}
\usepackage{fancyhdr,wrapfig,xcolor,graphicx,xparse,fontspec}
\usepackage[finnish]{babel}
\usepackage[%
    %a5paper,
    papersize={5.5in,8.5in},
    margin=0.75in,
    top=0.75in,
    bottom=0.75in,
    %twoside
    ]{geometry}

\newcounter{stepnum}

%% |=====8><-----| %%


\makeatletter

%% From Donald Arseneau. Add after the wrapping text. Whew!
\def\wrapfill{% Just glad it works.
    \par
  \ifx\parshape\WF@fudgeparshape
    \nobreak
    \ifnum\c@WF@wrappedlines>\@ne
      \advance\c@WF@wrappedlines\m@ne
      \vskip\c@WF@wrappedlines\baselineskip
      \global\c@WF@wrappedlines\z@
    \fi
    \allowbreak
    \WF@finale
  \fi
}


%% Used for the headnote and in \showit
%% If the text is small it is placed on one line;
%% otherwise it is put into a raggedright paragraph.
\long\def\testoneline#1{%
  \sbox\@tempboxa{#1}%
  \ifdim \wd\@tempboxa <0.75\linewidth
        \begingroup
            #1\par
        \endgroup
  \else
    \parbox{0.75\linewidth}{\raggedright#1}%
    \par
  \fi
}

\newif\if@mainmatter \@mainmattertrue

%% Borrowed from book.cls
\newcommand\frontmatter{%
    \cleardoublepage
  \@mainmatterfalse
  \pagenumbering{roman}}
\newcommand\mainmatter{%
    \cleardoublepage
  \@mainmattertrue
  \pagenumbering{arabic}}
\makeatother

%% Vary the colors at will

\definecolor{vegcolor}{rgb}{0,0.5,0.2}
\definecolor{frzcolor}{rgb}{0,0.8,0.8}
\definecolor{dessertcolor}{rgb}{0.5,0.2,0.1}
\definecolor{makeaheadcolor}{rgb}{0.5,0.5,0.6}

%% Thanks to alephzero for the excellent start:
%% #1 [optional headnote]; #2 Title of recipe; #3 [Initial instructions]
\NewDocumentCommand{\recipe}{o m o}{%
    \setcounter{stepnum}{0}%
    \newpage
    \thispagestyle{fancy}
    \lhead{}%
    \chead{}%
    \rhead{}%
    \lfoot{}%
    \rfoot{}%
    \begin{center}
    \subsection*{#2}%
    \end{center}
    \IfNoValueF{#1}{\begin{center}\testoneline{#1}\end{center}}
    \IfNoValueF{#3}{\noindent \textbf{\emph{Vinkki:}} \emph{#3}\par\medskip}
    \smallskip
}
\newcommand{\serves}[2][Annosta]{%
    \chead{#1 #2}}
\newcommand{\dishtype}[1]{%
    \rhead{#1}%
}
\newcommand{\dishother}[1]{%
    \lhead{#1}%
}
\newcommand{\vegetarian}{%
    {\large\color{vegcolor}\textbf{V}}%
}
\newcommand{\freeze}{%
    {\large\color{frzcolor}\textbf{F}}%
}
\newcommand{\dessert}{%
    {\large\color{dessertcolor}\textbf{D}}%
}
\newcommand{\makeahead}{%
    {\large\color{makeaheadcolor}\textbf{M}}%
}
%% Optional arguments for alternate names for these:
\newcommand{\preptime}[2][Valmistelu aika]{%
    \lfoot{#1: #2}%
}
\newcommand{\cooktime}[2][Valmistus aika]{%
    \rfoot{#1: #2}%
}
\newcommand{\oventime}[2][Uunissa]{%
    \rfoot{#1: #2}%
}
\newcommand{\temp}[1]{%
    #1°C}
%% Optional argument is the width of the graphic, default = 1in
\newcommand{\showpic}[3][1in]{%
    \begin{center}
        \bigskip
            \includegraphics[width=#1]{#2}%
            \par
            \medskip
            \testoneline{#3}%
            \par
    \end{center}%
}

\def\ucit#1{\uppercase{#1}}
\begingroup
    \lccode`~=`\^^M
    \lowercase{%
\endgroup%% Ingredient first, then measure; empty measure and/or unit = " . "
    %% *=column break; amount<space>ingredient
    \NewDocumentCommand{\ing}{u{ } u{~}}{% %% basically the same as: \def\ing#1 #2~{% requires xparse
        \noindent
        \if.#1% Is a heading, a non-ingredient, in the ingredients block
            \emph{#2}~ % A heading
        \else % Amounts containing spaces <1 teaspoon> have to use '~' <1~teaspoon>
            \textbf{\ucit#2, }#1~ %
        \fi
    }%
}%

\NewDocumentEnvironment{step}{}{%
    \parindent0pt
    \leftskip0pt
    \begin{minipage}{\textwidth}
        \begin{wrapfigure}{r}{0pt}
            \kern-0.5em
            \vrule width 1pt\enskip
            \begin{minipage}{0.5\textwidth}
                \leftskip=1.5em
                \parindent=-1.5em
                \parskip=0.25em
                \obeylines
                    \everypar={\ing}
}{%
        \wrapfill
    \end{minipage}
    \medskip
}

\NewDocumentCommand{\method}{}{%
            \end{minipage}
        \end{wrapfigure}
        \rightskip0pt plus 2em
        \parskip0.25em
        \everypar={\llap{\stepcounter{stepnum}\hbox to 1.5em{\thestepnum.\hfill}}}
}

\setmainfont{FreeSerif}

\pagestyle{plain}
\setlength{\intextsep}{0pt}

\begin{document}

\frontmatter
\tableofcontents

\mainmatter

\section{Liha}
\recette{Lihapullat ruotsalaisittain}

\begin{center}
\preptime{15 min} \cooktime{15 min} \people{4}
\end{center}

\recipebody{Ruotsalaiset lihapullat maistuvat kaikille, takuu varma arjen piristys. Tarjoillaan yleensä perunamuusin, ruskean kermakastikkeen ja puolukoiden kanssa.}

\recipe{
    Naudan jauhelihaa & \unit[400]{g}\\
    Korppujauhoja & \unit[25]{g}\\
    Maitoa & \unit[0,5]{dl}\\
    Cayanne Pippuria & \unit[1/4]{tl}\\
    Sipulijauhetta & \unit[1]{tl}\\
    Valkosipulijauhetta & \unit[1]{tl}\\
    Maustepippuria & \unit[1]{tl}\\
    Muskottipähkinää & \unit[1]{tl}\\
    Suolaa & \unit{maun mukaan}\\
    Pippuria & \unit{maun mukaan}\\
    }{
    \item Liota korppujauhot maidossa
    \item Sekoita keskenään liha, mausteet ja korppujauhot taikinaksi
    \item Pyörittelet taikinasta pienehköjä palloja
    \item Paista voissa keskisuurella lämmöllä
}

\info{Lihapullat voi kypsentää myös uunissa (\SI{225}{\celsius} 15min). Tällöin pullista tulee kevyempiä, mutta rapeaa kuorta ei synny.}

\recipe[Bolognesekastike on tomaattinen lihakastike, jota voi syödä sellaisenaan esimerkiksi pastan kanssa.]{Bolognesekastike}[Lisämakua kastikkeeseen voi saada lisäämällä joukkoon selleriä tai kananmaksaa. Valkoviinin voi korvata vedellä ja valkoviinietikalla.]
\serves{12}
\preptime{1h}
\cooktime{3,5h}

\begin{step}
  400g porkkanaa
  2kpl sipulia
  800g naudan jauhelihaa
  3rkl tomaattipyrettä
  750ml valkoviiniä
  1kg tomaattimurskaa
  1,5tl oreganoa
  1,5tl timjami
  1,5tl rosmariini
  2tl valkosipulijauhe
  2rkl balsamiviinietikkaa


\method
Raasta porkkanat ja sipulit ja kullota kattilassa

Siirrä kullotetut vihannekset syrjään

Ruskista jauheliha kattilassa perusteellisesti

Lisää joukkoon tomaattipyre ja sekoita ilman että pyre palaa

Lisää joukkoon valkoviini ja vihannekset

Lisää tomaattimurska

Anna kastikkeen kypsyä pienellä lämmöllä 2 tuntia välillä sekoittaen ettei se pala pohjaan

Kahden tunnin jälkeen lisää joukkoon yrtit, suola, pippuri ja balsamiviinietikka

Anna porista vielä tunti

\end{step}

\recipe[]{Lasagne}[Pastalevyjen käytössä kannattaa olla säästeliäs, sillä josta levyjä laittaa liikaa jää lasagne kuivaksi.]
\serves{12}
\preptime{5h}
\cooktime{1h}

\begin{step}
. Juustokastike:
  65 voita
  1kpl valkosipuli (kokonainen)
  50 vehnäjauhoja
  1l maitoa
  100g Pecorino Romano juustoa

\method
Hienonnan valkosipuli

Sulata voi pannulla

Lisää voin sekaan jauhot ja sekoita Roux:si

Lisää vähitelleen roux'n sekaan maito

Kun kaikki maito on lisätty, ota pannu pois levyltä ja anna jäähtyä niin ettei se enää kupli

Raasta juusto hieman jäähtyneen kastikkeen joukkoon

Sekoita mahdollisimman tasaiseksi

Maista suola ja lisää tarvittaessa. Kastikkeen pitäisi olla liian suolaista.
\end{step}

\begin{step}
. Lasagne
  15kpl Lasagnelevyjä
  1,5l Bolognesekastiketta (kts. edellinen resepti)
  1l Juustokastitta
  150g mozzarella juustoraastetta
\method
Voitela paistoastia kevyesti öljyllä

Asettele lasagnelevyistä ensimmäinen kerros astian pohjalle

Rakenna lasagne vuorotellen: bolognesekastike – lasagnelevyt – juustokastike – lasagnelevyt

Jatka kunnes astia tulee täyteen tai kastikeet loppuvat

Ylimmäksi kerrokeksi pitäisi tulla lasagnelevyjä, joiden päällä on kevyesti jäljelle jääviä kastikkeita

Paista lasagne \temp{200} uunissa 20 minuuttia

Levitä mozzarella tasaisesti lasagnen päälle

Paista lasagnea vielä toiset 20 minuuttia \temp{200}

\end{step}


\newpage
\section{Kana}
\recipe[Korealainen paistettu kana on helppo ja varsin nopea herkku isommallekkin joukolle. Perinteisesti KFC kanat paistetaan kahdesti. Ensimmäisellä kerralla leivite asetetaan paikoilleen ja toisella leivite ruskistetaan kullan ruskeaksi. Näin varmistetaan etteivät kanat ole rasvaisia vaan ihanan rapeita.]{Korean Fried Chicken}[Paistileikkeen voi korvata myös fileellä, tällöin paistoaikaa kannaattaa laskea. Uppopaistoöljy kannattaa puhdistaa ja kerätä talteen seuraavaa paistokertaa varten.]
\serves{6}
\preptime{30 min}
\cooktime{15 min}

\begin{step}
  800g Kanan paistileike
  350g Jauhoja
  350g Maissitärkkelystä
  15g Suolaa
  10g Leivinjauhetta
  8g Sipulijauhetta
  8g Valkosipulijauhetta
  \method
  Leikkaa lihat sopivan kokoisiksi paloiksi (noin 4 palaa per leike)

  Kuivaa kanat huolellisesti ennen leivittämistä

  Sekoita jauhot, tärkkelys, suola, leivin-, sipuli-, ja valkosipuliajauhe keskenään

  Siirrää noin $\frac{1}{3}$ kuivasta taikinasta toiseen astiaan

  Lisää tähän $\frac{1}{3}$ osaan vettä niin että koostumus vastaa lättytaikinaa

  Lisää märkää taikinaa nyt hieman kuivien aineiden sekaan ja sekoita nostellen

  Sekoita kunnes kuivasta taikinasta muodostuu kevyitä taikina paakkuja. Jos taikina näyttää jauhoiselta lisää hieman märkää taikinaa.

  Leivitä kanat hyvin kevyesti vehnäjauholla

  Leivitä jauhoiset kanat märällä taikinalla

  Lopuksi leivitä kanat paakkuisella kuivataikinalla

  Uppopaista leivitetyt kanat \temp{175} noin 7 minuuttia, kunnes leivite on asettunut

  Uppopaista kanat uudelleen \temp{190} noin 5 minuuttia, kunnes kanat ovat kauniin ruskeita
\end{step}

\recipe[Japanilainen curry on hyvää. Perinteisesti Japanilainen curry tarjoillaan riisin kanssa]{Japanilainen kana curry}[Curryyn sekaan voi laittaa mitä tahansa juureksia kaapista sattuu löytymään. Makeutukseen voi omenan sijasta laittaa esimerkiksi banaanin.]

\preptime{30 min}
\cooktime{1 h}
\serves{6}

\begin{step}
  . Pohja:
  1~kpl sipuli
  15~g inkivääriä
  4~kynttä valkosipulia
  1/2~dl vettä
  1/16~rkl ruokasoodaa
  1~kpl pieni porkkana
  1~rkl tomaattipyrettä
  1~tl suolaa
  1~kpl omena
  12~g curry jauhetta
  1/2~dl Kanalientä
  200~g porkkanaa
  200~g perunaa
  1~rkl soijakastiketta
  2-palaa tummaa suklaata
  \method
  Sekoita yhteen vesi ja ruokasooda

  Leikkaa kana sopivan kokoisiksi paloiksi

  Ruskista kana keskilämmöllä

  Raasta inkivääri ja valkosipuli ja pieni porkkana

  Poista kana kattilasta ja lisää valkosipulia ja inkivääri

  Ennenkuin pohja palaa lisää joukkoon sipulit ja ruokasoodaneste. Laita kansi päälle ja höyrytä

  Leikkaa juurekset isoiksi paloiksi

  Kun sipuli on höyrytetty sekoita ja anna sipulin karamellisoitua kunnolla

  Lisää raastettu porkkana

  Kun seos on karemellisoitunut lisää joukkoon curryjauhe ja tomaatti pyree ja anna paahtua

  Ennen kuin seos palaa lisää joukkoon kanaliemi ja sekoita sauvasekoittimella homogeeniseksi

  Raasta omena

  Lisää raastettu omena ja soijakastike

  Lisää kana ja juurekset

  Anna kypsyä välillä sekoittaen ettei kastike pala pohjaan

  Kun koostumus on paksua lisää suklaa

  Anna kypsyä noin tunti

\end{step}


\newpage
\section{Kasvis}
\recipe[Itsetehty tofu on hauska ja halpa vaihtoehto kaupasta ostetulle tofulle.]{Tofu}[Mikäli soijamaito ei konjaguloidu käytä enemmän etikkaa.]
\serves{4}
\preptime{12h}
\cooktime{1h}

\begin{step}
300g kuivattuja soijapapuja
30ml riisiviinietikkaa

\method
Turvota soijapapuja 1200ml vettä ainakin 12 tuntia

Lisää soijapavut ja vesi tehosekoittimeen ja sokoita mahdollisimman hienoksi. Hienonnettua seosta kutsutaan Namagoksi.

Lisää namago kiehuvaan 1600ml vettä ja kiehuta ainakin 10 minuuttia sekoittaen ettei namago pala pohjaan. Pinnalle nouseva vaahto kannattaa siivilöidä pois.

Siivilöi neste siivilällä ja huokella kankaala, niin että soijamaito valuu kattilaan ja soijapapujen kuidut (okara) jäävät kankaaseen. Okara on terveellistä ja sitä voi käyttää esimerkiksi vohveleissa.

Lämmitä jäljelle jäänyt soijamaito \temp{75} ja lisää riisiviinietikka tasaisesti maittoon ja sekoita kevyesti

Ota maito pois lämmöltä ja anna paakkujen muodostua noin kymmenen minuuttia

Siivilöi paakut muottiin ja valuta ylimääräinen neste pois.

Anna jäähtyä ja laita tofu kylmään veteen odottamaan käyttöä.

\end{step}

\recipe[Falafelit ovat hyviä. Nam]{Falafelit}[Paistoöljy kannattaa puhdistaa ja kerätä talteen]
\serves{8}
\preptime{24 h}
\cooktime{60 min}

\begin{step}
  5dl Kuivia kikherneitä
  1/2~tl Ruokasooda
  3dl Persiljaa
  1dl Tilliä
  1~kpl Sipulia
  8~kynttä Valkosipulia
  1~rkl Pippuria
  1~rkl Juustokuminaa
  1~rkl Jauhettua korianteria
  1~tl Cayenne pippuria
  1~tl Leivinjauhetta
  2~rkl Seesamin siemeniä
  \method
  Päivä ennen falafelien tekemistä laita kikherneet ja ruokasooda veteen turpoamaan, niin että herneet peittivät noin 5 cm vedellä. Turpoamisen jälkeen kuivaa herneet.

  kikherneet, yrtit, sipulit ja mausteet monitoimikoneen terällä hienoksi falafel taikinaksi.

  Peitä taikina ja siirrää jääkaapiin kiinteytymään ainakin tunniksi.

  Juuri ennen paistamista sekoita taikinan joukkoon leivinjauhe ja seesaminsiemenet.

  Muotoile käsin tai jäätelökauhalla taikinasta palloja.

  Uppopaista pallot noin 165 C / 325 F. Koska falafelit ovat pieniä öljyä ei tarvita paljoa
\end{step}

\recetter{Pinaattiletut}

\begin{center}
  \preptime{5 min} \cooktime{30 min} \people{6}
\end{center}

\recipebody{Nam nam vihreitä lettuja}

\recipe{
    Munia & \unit[3]{kpl}
    Maito & \unit[5]{dl}
    Vehnäjauhoja & \unit[3]{kpl}
    Suolaa & \unit[1]{tl}
    Pinaattia & \unit[150]{g}
    Öljyä & \unit[1]{rkl}
}{
    \item Silppua pinaatti
    \item Riko munat kulhoon
    \item Sekoita ainekset keskennän
    \item Anna turvota noin 30 minuuttia
    \item Paista letut
}

\info{Letut kannattaa tarjoilla puollukoiden ja raejuuston kanssa}

\recipe[Linssi keitto maistuu myös paatuneimmalle lihansyöjälle, kun sen tarjoilee leivän kanssa kylmänä talvipäivänä.]{Linssikeitto}[Keitosta saa täysin vegaanisen jos voin korvaa samalla määrällä esimerkiksi kookosöljyä]
\serves{6}
\preptime{10}
\cooktime{40}

\begin{step}
2~rkl voita
2~kpl sipulia
5~kpl valkosipulinkynttä
2~tl kurkumaa
1~tl inkiväärijauhetta
1~tl cajun maustetta
400g tomaattimurskaa
400g chilitomaattimurskaa
1,1~l vettä
3~dl punaisia linssejä (kuivana)
2~dl kookoskermaa
\method
Huuhdo linssit kylmällä vedellä huolellisesti siivilässä.

Hienonna sipulit ja valkosipuli ja kullota kattilassa, mutta älä päästä ruskistumaan

Sekoita mukaan mausteet ja anna hieman paahtua

Lisää joukkoon vesi ja tomaattimurskat

Anna kiehua muutama minuutti ja lisää joukkoon huuhdotut linssit

Anna keiton kiehua 30 minuuttia

Lisää kerma

Anna hautua 10 minuuttia

Maista suola
\end{step}


\newpage
\section{Jälkiruoat}
\recipe[Allekirjoittaneen suosikki täytekakku]{Kinuskikakku}[Kermavaahdon vatkaamiseen kannattaa käyttää pakastinkylmiä vispilöitä.Kakkupohja kannaattaa tehdä päivä ennen syömistä.]
\preptime{1 h}
\oventime{40 min}
\serves{16}

\begin{step}
  . Kakkupohja:
  4~kpl kanamunaa
  1~1/2~dl sokeria
  1~dl vehnäjauhoja
  1~dl perunajauhoja
  1~tl leivinjauhetta
  \method
  Vatkaa huoneenlämpöiset munat ja sokeri paksuksi vaahdoksi

  Yhdistä jauhot ja leivinjauhe ja siivilöi ne vaahdon sekaan.

  Sekoita tasaisesti nostellen

  Kaada taikina vuokaan

  Paista kakkupohja \temp{175} kypsäksi 35-40 min

  Kumoa hieman jäähtynyt kakkupohja
\end{step}

\begin{step}
  . Kostutus:
  1~dl Jaffaa tai appelsiini mehua
\method
Leikkaa jäähtynyt kakkupohja kolmeen osaan

Kostuta levyt Jaffalla
\end{step}

\begin{step}
  . Täytteet ja vaahto
  2~kpl banaania
  200~g vadelmia tai mansikoita
  3,3~dl kuohukermaa
  1,5~rkl sokeria
  1~tl vanilijasokeria
  \method
  Vatkaa kerma ja sokeri vaahdoksi

  Mausta kerma vanilijasokerilla

  Soseuta banaanit haarukalla

  Levitä kakun täyteväleihin ensin banaanisosetta, sitten marjoja ja lopuksi noin 1 cm:n kerros kermavaahtoa

  Nosta lopuksi kansi paikalleen ja anna kakun vetäytyä jääkaapissa vähintään 1h. Säästä ylimääräinen kermavaahto koristelua varten
\end{step}

\begin{step}
  . Kinuskikuorrutus
  1,5~dl Kuohukermaa
  1,25~dl fariinisokeria
  \method
  Keitä kermaa ja fariinisokeria kattilassa 5-7 min välillä sekoittaen, ettei seos kuohu yli

  Valuta pisaroita täytekakun päälle. Jos pisarat imeytyvät ei kinuski ole vielä tarpeeksi sakeaa

  Kun kinuski ei enää imeydy ota kattila pois levyltä ja anna kinuskin hieman jäähtyä

  Kaada jäähtynyt kinuski kakun päälle

  Levitä kinuskikuorrutus pyörittelemällä kakkua hieman

  Levitä ylimääräinen kermavaahto kakun reunoille
\end{step}

\recipe[]{Cinnabonit}[]
\serves{12}
\preptime{2,5 h}
\oventime{20 min}

\begin{step}
  . Taikina:
  580g jauhoja
  110g sokeria
  6g suolaa
  85g voita(suolaamatonta)
  235g täysmaitoa
  7g kuivahiivaa
  3~kpl kananmunia
  \method
  Sekoita yhteen jauhot, sokeri ja suolaa

  Lisää joukkoon pehmennetty voi ja sekoita yhteen

  Lämmitä maito kädenlämpöiseksi ja sekoita joukkoon hiiva

  Yhdistä jauhot, maito 2 kokonaista munaa ja yksi keltuainen

  Sekoita yhteen

  Vaivaa taikinaa hetki jauhotetulla pinnalla

  Anna taikinan nousta voidellussa astiassa 1-1,5 tuntia
\end{step}

\begin{step}
  . Täyte:
  225g fariinisokeria
  15g muscvado sokeria (vaihtoehtoinen)
  17g kanelia
  \method
  Sekoita yhteen fariinisokeri, muscvado sokeria ja kaneli
\end{step}

\begin{step}
  . Pullat:
  60g voita
  \method
  Kun taikina on noussut kauli se suorakulmion muotoon

  Voitele pinta pehmeällä voilla

  Ripottele täyte tasaisesti

  Rullaa tiukaksi rullaksi ja leikkaa tasaisen paksuisiksi pulliksi

  Voitela uuniastia voilla ja anna pullien nousta siinä noin 30-45 minuuttia

  Paista \temp{190} 15-20 minuuttia

  Anna jäähtyä 15 minuuttia ennen kuorruttamista
\end{step}

\begin{step}
  . Kuorrute:
  150g tuorejuustoa
  90g tomusokeria
  50ml täysmaitoa
  8g vaniljasokeria
  \method
  Vatkaa sulatejuusto tasaiseks

  Lisää joukkoon tomusokeri ja vaniljasokeri ja sekoita

  Lisää joukkoo maito ja sekoita

  Kuorruta jäähtyneet pullat
\end{step}

\recipe[]{Mokkapalat}[]
\preptime{20 min}
\oventime{15 min}
\serves{12}

\begin{step}
    . Pohja:
    4~kpl kananmunaa
    4~dl sokeria
    6~dl vehnäjauhoja
    4~tl leivinjauhetta
    4~tl vaniljasokeria
    2~dl maitoa
    200g sulatettua voita
    \method
    Vaahdosta munat ja sokeri

    Siivilöi ja sekoita keskenään jauhot, leivinjauhe ja vaniljasokeria

    Sekoita munien sekaan vuorotellen jauhoja ja maitoa

    Lisää lopuksi sulatettu voi ja sekoita tasaiseksi

    Paista \temp{200} uunissa noin 15 minuuttia

    Anna jäähtyä noin 15 minuuttia ennen kuorruttamista
\end{step}

\begin{step}
    . Kuorrute:
    75~g voita
    0,75~dl vahvaa kahvia
    240~g tomusokeria
    0,5~dl kaakaojauhetta
    2~tl vaniljasokeria
    \method
    Sulata voi kattilassa

    Lisää kahvi

    Siivilöi muut aineet siivilän läpi.

    Sekoita tasaiseksi

    Kaada hieman lämmin kuorrutus pohjan keskelle. (Kuumana kuorrute on liian löysää levitettäväksi ja silloin se imeytyy pohjaan. Kuorrutteen jäähtyessä, se paksuuntuu sopivaksi.)

    Levitä tarvittaessa lastalla reunoille

    Koristele ennen kuin kuorrutus kovettuu
\end{step}

\recipe[Hieman erikoisempi versio perinteisestä Suomalaisesta pannukakusta]{Lapin pannukakku}[]

\preptime{15 min}
\oventime{20 min}
\serves{16}

\begin{step}
  7~dl maitoa
  2~dl sokeria
  1~tl suolaa
  4~dl jauhoja
  4~kpl kanamunaa
  100~g voita
  \method
  Sekoita yhteen maito, sokeri, suoja ja jauhot

  Lisää kanamunat ja sekoita tasaiseksi

  Sulata voi suoraa paisto pellille

  Kaada seos pellille ja paista \temp{225} kunnen pannukakku on kypsä
\end{step}

\recipe[Vohvelit voi tarjoilla makeana hillon ja kermavaahdon kera, tai suolaisena esimerkiksi lohella ja salaatilla]{Vohvelit}[]

\serves{2}
\preptime{25min}
\cooktime{20min}

\begin{step}
  2~kpl kananmunaa
  4dl maitoa
  3d vehnäjauhoja
  0,5tl leivinjauhetta
  1tl suolaa
  30g sulatettua voita
\method
Vatkaa munien rakenne rikki ja sekoita joukkoon maitoa

Siivilöi sekaan jauhot ja leivinjauhe

Lisää joukkoon suola ja voi

Anna taikinan turvota 15 minuuttia

Lämmitä vohvelirauta ja voitele voilla

Lisää 1,5 kauhallista taikinaa vohvelirautaan ja paista kunnen vohveli on kullanruskea

Tarjoile välittömästi
\end{step}


\newpage
\section{Kastikkeet}
\recette{Caesar kastike}

\begin{center}
  \preptime{5 min} \cooktime{5 min} \people{4}
\end{center}

\recipebody{Voimakas Caesar kastike on ehdoton osa hyvää caesar salaattia.}

\recipe{
  Valkosipulia & \unit[3]{kpl}\\
  Sinappia & \unit[2]{tl}\\
  Worcester kastiketta & \unit[1]{tl}\\
  Sitruunan mehua & \unit[2]{tl}\\
  Punaviinietikkaa  & \unit[1,5]{tl}\\
  Öljyä  & \unit[0,5]{dl}\\
  Suolaa  & \unit[0,5]{tl}\\
  Pippuria  & \unit{maun mukaan}\\
}{
    \item Purista valkosipulit puristimella tai leikkaa veitsellä niin pieneksi kuin mahdollista
    \item Sekoita ainekset keskenään
    \item Siirrä kylmään odottamaan käyttöä
}
\info{Kastike on rikasta, joten lisää aluksi vain vähän. Kastike kannattaa tehdä useita tunteja ennen käyttöä, jotta maut kerkeävät tasoittua ja raa'an valkosipulin maku pehmenee.}

\recette{Katsutyylinen dippikastike}
\begin{center}
\preptime{15 min} \cooktime{15 min} \people{4}
\end{center}

\recipebody{Maukas dippi kastike kanakatsun kanssa.}

\recipe{
    Ketsuppia & \unit[1]{dl}\\
    Worchestershire-kastiketta & \unit[0.5]{dl}\\
    Sakea & \unit[0.5]{dl}\\
    Sokeria & \unit[0.25]{dl}\\
    Soijakastiketta & \unit[1]{rkl}\\
    Miriniä & \unit[0.25]{dl}\\
    Valkosipuliakynsi & \unit[2]{kpl}\\
    Inkivääriä & \unit[]{Yhtä paljon kuin valkosipulia}\\
    }{
    \item Kuori ja raasta valkosipulit sekä inkivääri niin hienoksi, kuin mahdolllista
    \item Sekoita kaikki ainekset kattilaan
    \item Anne sakeutua matalalla lämmöllö 20 minuutin ajan, koko ajan sekoittaen
    \item Siirrä kylmään odottamaan käyttöä
}


\newpage
\section{Juomat}
\recipe[Maukas aeropress kahvi James Hoffmanin tapaan.]{Aeropress Kahvi - James Hoffman}[Kahvi ja vesi kannattaa mitata keittövaa'alla, jotta kahvi on joka kerralla samanlaista. Veden lämpötilaa kannattaa hieman laskea (noin 5 astetta), mikäli käytössä on tumma paahtoinen kahvi. Aeropressillä tehty kahvi muistuttaa maultaan hieman espressoa, joten valmista juomaa voi jatkaa hieman kuumalla vedellä, jolloin se vastaa enemmän perinteistä suodatinkahvia.]
\preptime{1 min}
\cooktime{3 min}
\serves{1}

\begin{step}
  12g Vasta jauhettua kahvia
  200g \temp{99} vettä
  \method
  Jauha haluamasi kahvipavut keskihienoksi

  Laita aeropressiin hautumaan 2 minuutiksi

  Kevyesti pyöritä pressiä

  Anna hautua vielä 30 sekuntia

  Paina pressi kevyesti pohjaan. Noin 30 sekuntia.
\end{step}



\end{document}
