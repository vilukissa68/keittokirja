\recette{Korean Fried Chicken}

\begin{center}
\preptime{30 min} \cooktime{15 min} \people{6}
\end{center}

\recipebody{Korealainen paistettu kana on helppo ja varsin nopea herkku isommallekkin joukolle. Perinteisesti KFC kanat paistetaan kahdesti. Ensimmäisellä kerralla leivite asetetaan paikoilleen ja toisella leivite ruskistetaan kullan ruskeaksi. Näin varmistetaan etteivät kanat ole rasvaisia vaan ihanan rapeita.}

\recipe{
  Kanan paistileike & \unit[800]{g}\\
  Jauhoja & \unit[350]{g}\\
  Maissitärkkelystä & \unit[350]{g}\\
  Suolaa & \unit[15]{g}\\
  Leivinjauhetta & \unit[10]{g}\\
  Sipulijauhetta & \unit[8]{g}\\
  Valkosipulijauhetta & \unit[8]{g}\\
    }{
    \item Leikkaa lihat sopivan kokoisiksi paloiksi (noin 4 palaa per leike)
    \item Kuivaa kanat huolellisesti ennen leivittämistä
    \item Sekoita jauhot, tärkkelys, suola, leivin-, sipuli-, ja valkosipuliajauhe keskenään
    \item Siirrää noin $\frac{1}{3}$ kuivasta taikinasta toiseen astiaan
    \item Lisää tähän $\frac{1}{3}$ osaan vettä niin että koostumus vastaa lättytaikinaa
    \item Lisää märkää taikinaa nyt hieman kuivien aineiden sekaan ja sekoita nostellen
    \item Sekoita kunnes kuivasta taikinasta muodostuu kevyitä taikina paakkuja. Jos taikina näyttää jauhoiselta lisää hieman märkää taikinaa.
    \item Leivitä kanat hyvin kevyesti vehnäjauholla
    \item Leivitä jauhoiset kanat märällä taikinalla
    \item Lopuksi leivitä kanat paakkuisella kuivataikinalla
    \item Uppopaista leivitetyt kanat $175^\circ$C noin 7 minuuttia, kunnes leivite on asettunut
    \item Uppopaista kanat uudelleen $190^\circ$C noin 5 minuuttia, kunnes kanat ovat kauniin ruskeita
}

\info{Paistileikkeen voi korvata myös fileellä, tällöin paistoaikaa kannaattaa laskea. Uppopaistoöljy kannattaa puhdistaa ja kerätä talteen seuraavaa paistokertaa varten.}
