\recipe[Korealainen paistettu kana on helppo ja varsin nopea herkku isommallekkin joukolle. Perinteisesti KFC kanat paistetaan kahdesti. Ensimmäisellä kerralla leivite asetetaan paikoilleen ja toisella leivite ruskistetaan kullan ruskeaksi. Näin varmistetaan etteivät kanat ole rasvaisia vaan ihanan rapeita.]{Korean Fried Chicken}[Paistileikkeen voi korvata myös fileellä, tällöin paistoaikaa kannaattaa laskea. Uppopaistoöljy kannattaa puhdistaa ja kerätä talteen seuraavaa paistokertaa varten.]
\serves{6}
\preptime{30 min}
\cooktime{15 min}

\begin{step}
  800g Kanan paistileike
  350g Jauhoja
  350g Maissitärkkelystä
  15g Suolaa
  10g Leivinjauhetta
  8g Sipulijauhetta
  8g Valkosipulijauhetta
  \method
  Leikkaa lihat sopivan kokoisiksi paloiksi (noin 4 palaa per leike)

  Kuivaa kanat huolellisesti ennen leivittämistä

  Sekoita jauhot, tärkkelys, suola, leivin-, sipuli-, ja valkosipuliajauhe keskenään

  Siirrää noin $\frac{1}{3}$ kuivasta taikinasta toiseen astiaan

  Lisää tähän $\frac{1}{3}$ osaan vettä niin että koostumus vastaa lättytaikinaa

  Lisää märkää taikinaa nyt hieman kuivien aineiden sekaan ja sekoita nostellen

  Sekoita kunnes kuivasta taikinasta muodostuu kevyitä taikina paakkuja. Jos taikina näyttää jauhoiselta lisää hieman märkää taikinaa.

  Leivitä kanat hyvin kevyesti vehnäjauholla

  Leivitä jauhoiset kanat märällä taikinalla

  Lopuksi leivitä kanat paakkuisella kuivataikinalla

  Uppopaista leivitetyt kanat \temp{175} noin 7 minuuttia, kunnes leivite on asettunut

  Uppopaista kanat uudelleen \temp{190} noin 5 minuuttia, kunnes kanat ovat kauniin ruskeita
\end{step}
