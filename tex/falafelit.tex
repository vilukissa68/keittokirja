\recette{Falafelit}

\begin{center}
\preptime{24 h} \cooktime{60 min} \people{4}
\end{center}

\recipebody{Falafelit ovat hyviä. Nam}

\recipe{
  Kuivia kikherneitä & \unit[5]{dl}\\
  Ruokasoodaa & \unit[0.5]{tl}\\
  Persiljaa & \unit[3]{dl}\\
  Tilliä & \unit[1]{dl}\\
  Sipulia & \unit[1]{kpl}\\
  Valkosipulia & \unit[8]{kynttä}\\
  Pippuria & \unit[1]{rkl}\\
  Juustokuminaa & \unit[1]{rkl}\\
  Jauhettua korianteria & \unit[1]{rkl}\\
  Cayenne pippuria & \unit[1]{tl}\\
  Leivinjauhetta & \unit[1]{tl}\\
  Seesamin siemeniä & \unit[2]{rkl}\\
}{
  \item Päivä ennen falafelien tekemistä laita kikherneet ja ruokasooda veteen turpoamaan, niin että herneet peittivät noin 5 cm vedellä. Turpoamisen jälkeen kuivaa herneet.
    \item Jauha kikherneet, yrtit, sipulit ja mausteet monitoimikoneen terällä hienoksi falafel taikinaksi.
    \item Peitä taikina ja siirrää jääkaapiin kiinteytymään ainakin tunniksi.
    \item Juuri ennen paistamista sekoita taikinan joukkoon leivinjauhe ja seesaminsiemenet.
    \item Muotoile käsin tai jäätelökauhalla taikinasta palloja.
    \item Uppopaista pallot noin 165 C / 325 F. Koska falafelit ovat pieniä öljyä ei tarvita paljoa
}
