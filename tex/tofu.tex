\recipe[Itsetehty tofu on hauska ja halpa vaihtoehto kaupasta ostetulle tofulle.]{Tofu}[Mikäli soijamaito ei konjaguloidu käytä enemmän etikkaa.]
\serves{4}
\preptime{12h}
\cooktime{1h}

\begin{step}
300g kuivattuja soijapapuja
30ml riisiviinietikkaa

\method
Turvota soijapapuja 1200ml vettä ainakin 12 tuntia

Lisää soijapavut ja vesi tehosekoittimeen ja sokoita mahdollisimman hienoksi. Hienonnettua seosta kutsutaan Namagoksi.

Lisää namago kiehuvaan 1600ml vettä ja kiehuta ainakin 10 minuuttia sekoittaen ettei namago pala pohjaan. Pinnalle nouseva vaahto kannattaa siivilöidä pois.

Siivilöi neste siivilällä ja huokella kankaala, niin että soijamaito valuu kattilaan ja soijapapujen kuidut (okara) jäävät kankaaseen. Okara on terveellistä ja sitä voi käyttää esimerkiksi vohveleissa.

Lämmitä jäljelle jäänyt soijamaito \temp{75} ja lisää riisiviinietikka tasaisesti maittoon ja sekoita kevyesti

Ota maito pois lämmöltä ja anna paakkujen muodostua noin kymmenen minuuttia

Siivilöi paakut muottiin ja valuta ylimääräinen neste pois.

Anna jäähtyä ja laita tofu kylmään veteen odottamaan käyttöä.

\end{step}
